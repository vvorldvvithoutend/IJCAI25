\documentclass{article}
\usepackage{a4wide}
\usepackage{amsthm, amsmath, logicproof, stmaryrd, amssymb, amsfonts}
\usepackage{multicol}
\usepackage{tikz}
\usepackage{macros}
\usepackage{mathrsfs}
\usetikzlibrary{shapes.misc, fit, decorations.pathreplacing,calligraphy,positioning} %for crossing-out arrows and rectangles around nodes
\usepackage{float}

\usepackage{soul}
\usepackage{todonotes}
\newcommand\rustam[1]{\todo[color=blue!30,size=\small,inline]{Rustam: #1}}
\newcommand\davide[1]{\todo[color=green!30,size=\small,inline]{Davide: #1}}

\theoremstyle{definition}
\newtheorem{definition}{Definition}[section]
\newtheorem{lemma}{Lemma}
\newtheorem{theorem}{Theorem}
\newtheorem{proposition}{Proposition}


\allowdisplaybreaks

\title{Let's axiomatise at least some fragment}

\date{\today}

\begin{document}
\maketitle

Given a finite, non-empty set $\Ag$ of \emph{agents}, and two countable disjoints sets $\V$ and $\Ap$ of, respectively, variables and atomic propositions (or atoms), formulae of Basic Strategy Logic ($\BSL$ for short) are inductively defined by the following grammar: 
$$\varphi := p \mid \neg \varphi \mid \ (\varphi \land \varphi) \mid \prox \varphi \mid \assign a x \varphi \mid \exists x \varphi \mid \forall x \varphi $$

\noindent where $p\in \Ap$, $a\in \Ag$ and $x\in \V$. 


\begin{definition}
      Given a formula $\varphi$, we define its set of free variables  and agents $\FV(\varphi)$ by the following cases: 

    \begin{enumerate}
        
        \item If $\varphi\in \Ap$ then $\FV(\varphi)=\emptyset$; 
        \item If $\varphi=\neg \varphi_1$ then $\FV(\varphi)=\FV(\varphi_1)$; 
        \item If $\varphi=\varphi_1 \land \varphi_2$ then $\FV(\varphi)=\FV(\varphi_1)\cup \FV(\varphi_2)$; 
        \item If $\varphi=\prox \varphi_1$ then $\FV(\varphi)=\FV(\varphi_1)\cup \Ag$;
        
        \item if $\varphi= Q x \varphi_1$ with $Q\in \set{\exists,\forall}$ then $\FV(\varphi)=\FV(\varphi_1)\setminus \set{x}$
        \item if $\varphi=\assign{a}{x}\varphi_1$ then 
        \begin{enumerate}
            \item if $a\notin \FV(\varphi_1)$ then $\FV(\varphi)=\FV(\varphi_1)$;
            \item else $\FV(\varphi)=\FV(\varphi_1)\setminus\set{a}$.
        \end{enumerate}
    \end{enumerate}
\end{definition}


\begin{definition}
       Let $\varphi$ be a formula and $x$ and $y$ be two variables. We define the result of the \emph{capture avoiding substitution} of every occurrence of $x$ by $y$ in $\varphi$, written $\varphi[y/x]$, by the following clauses: 

    $$\begin{array}{l@{\quad=\quad}l@{\qquad}l}
        p[y/x] & p &\mbox{ for any $p\in \Ap$}\\
       ( \circ\, \psi )[y/x] & \circ\, (\psi [y/x]) & \mbox{for $\circ\in \set{\neg,\prox}$} \\
       (\theta\land  \psi)[y/x] & \theta[y/x]  \land \psi[y/x] & \\
       (Q{x} \psi)[y/x] & Q{x}\psi\\
       (  Qv \psi) [y/x] &  Qv \psi \quad  &\mbox{if $x \notin \FV(\psi)$}\\
       (Qv \psi) [y/x] &  Qv (\psi [x/y]) & \mbox{if $x\in \FV(\psi)$}\\
       (Qy \psi)[y/x] & Qz ((\psi[z/y])[y/x])) &\mbox{if $x\in \FV(\psi)$}\\
       (\assign{a}{x} \psi)[y/x] & \assign{a}{y} \psi &\mbox{if $x\notin\FV{\psi}$} \\
       (\assign{a}{x} \psi) [y/x] & \assign{a}{y}(\psi[y/x]) & \mbox{if $x\in \FV(\psi)$}\\
       (\assign{a}{v} \psi) [y/x] & \assign{a}{v} \psi & \mbox{if $x\notin\FV(\psi)$}\\
       (\assign{a}{v} \psi) [y/x] & \assign{a}{v} (\psi[y/x]) & \mbox{if $x\in \FV(\psi)$}
    \end{array}$$

    
\end{definition}



The semantics of $\BSL$ formulas is specified with respect to concurrent games structures (CGSs for short). Intuitively, a CGS is a labeled directed  graph that represents the possible evolution of a given Multi-Agent System with respect to simultaneous choices of actions of a group of (autonomous) agents. 
Both states and edges are labeled by members of two disjoints alphabets. States are labeled by atomic propositions. These atomic propositions represent the properties that are true at a given state. Each edge is labeled by a tuple, and each member of a given tuple represents an action that is available for a given agent at the source state of the edge. 

\begin{definition}
Given a set $\Ag$ of agents, and a set $\Ap$ of atomic proposition, a CGS constructed over $\Ap$ and $\Ag$ is a tuple 
$\mathcal{G}=\tuple{S, \Ac,T,\mathcal{L}}$ where: 
\begin{itemize}
    \item $S$ is a countable non-empty set of states;
    \item $\mathsf{Ac}$ is a finite set of actions. We denote by $\mathcal{D}$ the set of tuples of actions whose length is $|\Ag|$
    Elements of $\mathcal{D}$ will be called \emph{decisions}; 
    \item   $T: S \times \mathcal{D} \to S$ is the transition function, mapping each pair $\tuple{s,d}$ composed of a state and a decision to a state $T(\tuple{s,d})$; 
    \item finally $\mathcal{L}: S\to 2^{\Ap}$ is the labeling function, which specifies the set of atomic propositions that are true at a given state; 
 \end{itemize}
\end{definition}


\begin{definition}
    Given a CGS $\G$, an assignment over $\G$ is a map 
     $\sigma : \V \cup \Ag \to \Ac$ sending each variable and each agent to  an action. If $\sigma$ is an assignment, $l\in \V \cup \Ag$ and $f\in \Ac$, we let $\sigma[f/l]$ denote the assignment $\sigma'$ such that $\sigma'(l')=\sigma(l')$ when $l'\neq l$ and $\sigma'(l)=f$. 
 
\end{definition}


\begin{definition}
     Given a CGS $\G$, a state $s$, an assignment $\sigma$ and a formula $\varphi$, the satisfaction relation $\G,s,\sigma \models \varphi$ is defined by induction on the structure of $\varphi$: 





Let $\Ag=\set{a_1,\ldots, a_n}$ be the set of agents. A binding prefix $\mathcal{B}$ is a sequence: 

$$\mathcal{B}^A = \assign{a_1}{x_1}\cdots \assign{a_n}{x_n}$$

where each $x_i$ is a variable. The set of formulas of Coalition Strategy Logic  is recursively defined by the following grammar

$$\varphi:= p \mid \neg \varphi \mid \varphi \land \varphi \mid  \mathcal{B}^A \prox \varphi\mid \forall x \varphi $$


The semantics of $\BSL$ formulas is specified with respect to concurrent games structures (CGSs for short). Intuitively, a CGS is a labeled directed  graph that represents the possible evolution of a given Multi-Agent System with respect to simultaneous choices of actions of a group of (autonomous) agents. 
Both states and edges are labeled by members of two disjoints alphabets. States are labeled by atomic propositions. These atomic propositions represent the properties that are true at a given state. Each edge is labeled by a tuple, and each member of a given tuple represents an action that is available for a given agent at the source state of the edge. 

\begin{definition}
Given a set $\Ag$ of agents, and a set $\Ap$ of atomic proposition, a CGS constructed over $\Ap$ and $\Ag$ is a tuple 
$\mathcal{G}=\tuple{S, \Ac,T,\mathcal{L}}$ where: 
\begin{itemize}
    \item $S$ is a countable non-empty set of states;
    \item $\mathsf{Ac}$ is a finite set of actions. We denote by $\mathcal{D}$ the set of tuples of actions whose length is $|\Ag|$
    Elements of $\mathcal{D}$ will be called \emph{decisions}; 
    \item   $T: S \times \mathcal{D} \to S$ is the transition function, mapping each pair $\tuple{s,d}$ composed of a state and a decision to a state $T(\tuple{s,d})$; 
    \item finally $\mathcal{L}: S\to 2^{\Ap}$ is the labeling function, which specifies the set of atomic propositions that are true at a given state; 
 \end{itemize}
\end{definition}


\begin{definition}
    Given a CGS $\G$, an assignment over $\G$ is a map 
     $\sigma : \V \cup \Ag \to \Ac$ sending each variable and each agent to  an action. If $\sigma$ is an assignment, $l\in \V \cup \Ag$ and $f\in \Ac$, we let $\sigma[f/l]$ denote the assignment $\sigma'$ such that $\sigma'(l')=\sigma(l')$ when $l'\neq l$ and $\sigma'(l)=f$. 
 
\end{definition}


\begin{itemize}
    \item $\G,s,\sigma \models p$ iff $p\in \mathcal{L}(s)$; 
    \item $\G,s,\sigma \models \neg \psi$ iff it is not the case that $\G,s,\sigma \models \psi$ (denoted $\G,s,\sigma \not\models \psi)$; 
    \item $\G,s,\sigma \models \theta \land\psi $ iff $\G,s,\sigma \models \theta$ and $\G,s,\sigma \models \psi$;
    \item $\G,s,\sigma \models \assign{a_1}{x_1}\cdots \assign{a_n}{x_n} \ \prox \varphi $ iff $\G,s' , \sigma' \models \psi $ where $s'=T(s, \Pi_{a\in \Ag}\sigma'(a))$ and $\sigma'=\sigma'=\sigma[\sigma(x_i)/a_i]$; 
    
    \item $\G,s,\sigma \models \forall x \psi$ iff for every  action $\alpha\in \Ac$ we have that $\G,s,\sigma[\alpha/x] \models \psi$.
    
\end{itemize}
    
\end{definition}





\end{document}
