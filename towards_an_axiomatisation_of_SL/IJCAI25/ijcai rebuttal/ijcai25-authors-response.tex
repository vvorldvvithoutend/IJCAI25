% IJCAI 2024 Author's response

% Template file with author's response

\documentclass{article}
\pdfpagewidth=8.5in
\pdfpageheight=11in
\usepackage{ijcai25-authors-response}


\usepackage{times}
\usepackage{soul}
\usepackage{url}
\usepackage[hidelinks]{hyperref}
\usepackage[utf8]{inputenc}
\usepackage[small]{caption}
\usepackage{graphicx}
\usepackage{amsmath}
\usepackage{amsthm}
\usepackage{booktabs}
\usepackage{algorithm}
\usepackage{algorithmic}
\usepackage{lipsum}
\urlstyle{same}
\usepackage{amsmath,amssymb}

\newtheorem{example}{Example}
\newtheorem{theorem}{Theorem}

\begin{document}
\paragraph{Reviewer 1} Thank you for such a thorough and superb review!

%Q1 The semantics given between line 320 and 321 does not take into account explicit
%actions in M; as it is, it is basically a rephrasing of CL. Am I missing
%something? If not, fix it.


\textbf{A1}. Observe that a modality marker is a prefix of size $|Agt| = n$, where each element is either a quantifier or an explicit action (the author of the corresponding papers considers in one case modal markers with explicit actions and in the other --- without. We took a more general definition for a fairer comparison). Then, in the semantics, we consider only $m \leqslant n$ existential quantifiers. You are right that this is similar to CL (and we believe that this was the intention of the author of the corresponding paper), but the difference is that some actions in the modality marker are chosen 'in advance' (and thus there is no corresponding quantification over them). Thus, quantification over action profiles $A$ in the definition of the semantics is over such action profiles where explicit actions of the given modality marker are fixed.  We will make this clearer in the final version and will provide an example of a modality marker.   

%Q2 t lines 378-380, you notice that quantifications in FOCL are first-order, while
%they are second-order in SL. What about the next-time fragment of SL? Are they
%first-order there?
%Once again, a comparison with the closest formalism is missing.

\textbf{A2}. Thank you for this question. We should have been more precise: in the fragment of SL where every 'next' operator is preceded by assignment operators, quantification over strategies is indeed unnecessary. A complete assignment of variables in SL determines a path through the corresponding strategies. For a single 'next' step, we only need to evaluate the second state of that path. This is equivalent to checking whether the tuple of actions generated by the strategies in that state leads to a successor state where the formula holds. This reasoning doesn't seem to hold if nested 'next' operators are allowed. In such cases, we believe that some form of quantification is required—if not over full strategies, then at least over finite sequences of decisions. 
We will clarify this aspect in the final version of the paper. 

Regarding a proper comparison, we would also like to point out that, to the best of our knowledge, the 'next-time fragment of SL' has never been officially introduced or mentioned in the literature. The idea to single out this fragment came to us as a result of the exploration of the satisfiability problem of FOCL.

%Q3 I do not see the derivation of item 3 of Proposition 5. And I do not see the
%sense of the whole formula. Can you please clarify this? Probably, it is a typo;
%a possible alternative could be
%$\exists z ( \phi[z/y] -> \forall y \phi ), where z \notin \phi
%Notice also "z \notin \phi" rater "z \notin FV(\forall y \phi)$, so that you can
%substitute free and non-free occurrences.

\textbf{A3}. Sorry, it is indeed a typo, and we will correct it as you suggest. Thank you ! It is nothing but the "drinker's theorem" of FOL. 

%Q4 In the proof of completeness of the axiomatization, shouldn't you also notice
%(and prove) that set of states $S^C$ in Def. 12 contains at most countably many
%states.

\textbf{A4}. We’ve revised the text, and now we only assume that the set of states of a model is non-empty as the set of maximally consistent sets is uncountable, i.e. it has the continuum cardinality. Thank you for noticing this aspect. 

%Q5 The proof of Lemma 7 only works for sentence, since the $\forall$-property, used
%at line 534, guarantees the existence of a constant a such that formula
%"$\psi[a/x] -> \forall x \psi$" belongs to X only for formulas $\psi$ such that
%"$\forall x \psi$" is a sentence.
%Can the proof be adapted easily? Otherwise, change the statement of Lemma 7.

\textbf{A5}. Yes, the proof can be adapted easily, but since we are working with sentences you are right that it is better to state "for every sentence". 


\textbf{A6 (Additional comments)}. 
Thank you for your feedback on improving the definitions, propositions, and lemmas. We will implement all of the suggested improvements. Here are answers to some particular suggestions. 

Lines 324-325: We will correct the wording of 'no matter' as you suggest. %\textbf{The Gen rule: ???} 

We agree that the wording of the proof of Lemma 6 can be improved so that the case $n=0$ aligns explicitly with Lemma 5. In fact what we are proving is that, for any $n$, $Z\cup \{\theta_n\}$ is consistent. Where $\theta_0=\varphi$  and $\theta_{n+1}= \theta_n \land \xi[a/x]\to \forall x \xi $.

Regarding lines 929–933: our semantics is defined only for sentences (satisfaction of open formulas is reduced to sentences via closure, as noted in Remark 7). Extending this to arbitrary formulas would require defining the semantics with respect to assignments, which we aimed to avoid.


\paragraph{Reviewer 2} Thank you for your review!
%Q1 Line 147 says that the proposition 1 follows from totality of frames. I agree that totality implies the proposition, but when is totality required? I only see the frames are serial and functional. 

\textbf{A1}. Sorry, it is a typo. We used the term "totality" but we meant "functionality". Seriality and functionality together grant us that there is exactly one outgoing edge for each decision at each state. 

%Q2 it seems that each time you are working on a particular class of models, in particular, with a fixed n (otherwise validity and other notions do not make sense). Is that the case? Moreover, can you generalize the classes of models, for e.g. the satisfiability problem?

\textbf{A2}. Yes, validity is given with respect to a given signature. If you mean models in which the relation is not necessarily serial and functional, then we would say that the generalisation must be possible, but it is unclear to us whether the obtained logic would be interesting for strategic reasoning. For instance, dropping the functionality requirement would mean that decisions could non-deterministically lead from a single state to multiple states of the model. However, we would like to point out that there is some recent work that explores CL with relaxed conditions on edges (recent papers involving Fengkui Ju). A similar analysis is beyond the scope of the current paper, but we will mention this as a future research direction in the final version.

\paragraph{Reviewer 3}
Thank you for such an eloquent review! We believe that FOCL has the potential in the area of specification and verification of MAS. This is supported by the fact that we can express Nash equilibrium (NE) in our logic. To take a classic warehouse robots scenario, assume that there are three robots that need to reach the same location. There are three paths to the location, and if two robots choose the same path, they collide and fail to fulfill their goals. Hence, in this scenario we can verify that for the given situation, it is indeed the NE for all three robots to take different paths. 

As another example, we can continue on the theme of smart contracts on a blockchain, and assume that two agents want to exchange their assets or private information, like passwords, without necessarily trusting each other. In such a situation, a smart contract will function as an escrow (realised in, e.g., \textit{atomic swap} smart contracts). First, we can use FOCL to verify that (some parts of) the smart contract fulfills its functionality in a safe manner. For example, that over a set finite number of rounds each agent receives the asset of the other and none of the agents holds the other's asset without giving up their own. Second, we can further extend the scenario with a third, malicious, agent that gained access to the communication channel and poses as one of the honest agents (the classic man-in-the-middle attack scenario). Now, having the ability to express strategy sharing, we can model the situation of the malicious agent mimicking one of the honest ones by executing the same actions. Having this in hand, we can then check whether a given (model of the) smart contract still fulfills at least some of the desirable functionality and safety requirements. 

We will write a paragraph on potential applications in the final version.

\textbf{Reviewer 4} Thank you!


\end{document}

